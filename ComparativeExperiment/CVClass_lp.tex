%!TEX encoding = UTF-8 Unicode

%---------------------------------------------------------------%
%  使用XeLaTex编译
%  参考文献的排版,请创建 .bib 文件, 并使用 BibTex 或者 Biber(有中文参考文献时) 进行排版。
%---------------------------------------------------------------%

%===============================================================%
%  固定模板,请不要修改这一部分

\documentclass[a4paper,11pt,onecolumn,UTF8]{article}
\usepackage{CVClassTemplate}                                  
\setmainfont{Times New Roman}               

\newcommand{\mysecondauthor}{null}
\newcommand{\mythirdauthor}{null} 
\newcommand{\myfourthauthor}{null}
\newcommand{\myfifthauthor}{null}
%===============================================================%


%===============================================================%
%  基本信息配置
%---------------------------------------------------------------%
\newcommand{\mytitle}          
{基于改进 SAM2 与时序平滑约束的超声舌面实时分割与追踪研究}

\newcommand{\myfirstauthor}        
{\kaishu [你的姓名]}

\newcommand{\myfirstaffiliation}   
{\small 清华大学计算机系}

\newcommand{\myfirstemail}         
{\small [你的学号]@mails.tsinghua.edu.cn}

\newcommand{\myabstract}{
超声舌面分割在语音科学研究及临床医学诊断中具有核心地位。然而,超声影像中特有的斑点噪声、边界模糊以及动态舌体运动的复杂性,给传统分割方法带来了严峻挑战。本文提出了一种基于改进 Segment Anything Model 2 (SAM2) 的超声舌面分割与追踪框架。针对超声边缘锯齿化问题,本文在全参数微调过程中引入了包含全变分损失 (Total Variation Loss) 的复合损失函数,显著提升了掩码的解剖学平滑度。同时,设计了一种结合 YOLO 检测框与亮度启发式分析的自动化提示点生成策略,实现了长序列视频的鲁棒追踪。实验表明,改进后的模型在测试集上的平均 IoU 达到 0.88,结合 B-Spline 样条拟合提取的舌中线具有极高的临床应用价值。
}

\newcommand{\mykeywords}{
计算机视觉;SAM2;超声图像分割;全变分损失;舌面追踪
}

%===============================================================%
\begin{document}

\printtitlepage

\section{概述}
超声成像技术凭借其高帧率、非侵入性的特点,成为观测舌体运动的主流手段。但在计算机视觉任务中,由于超声影像信噪比低,舌面边缘往往与背景组织回声交织。传统基于主动轮廓模型(Snakes)的方法易陷入局部最优,且对初始点敏感。近年来,基础模型如 SAM 的出现定义了交互式分割的新范式,但其在医疗特定领域的零样本性能仍受限。本文的研究动机在于利用双 RTX 4090 的算力优势,对最前沿的 SAM2 进行微调,旨在构建一个能自动提取“丝滑”舌面轮廓的实时系统。

\section{相关工作}
近年来,医学影像分割从传统的 U-Net 架构转向了基于 Transformer 的大模型。SAM2 通过引入记忆库(Memory Bank)机制,首次实现了高质量的视频目标分割(VOS)。然而,直接将 SAM2 应用于超声影像时,由于缺乏空间连续性约束,预测结果常出现不自然的颗粒感和边缘波动。本文通过引入空间平滑先验,弥补了通用大模型在特定医学任务上的短板。

\section{本文方法}
本文提出的系统架构由三个核心部分组成:

\subsection{自动化多模态提示策略}
为了摆脱对手工标注的依赖,本文实现了一个分层提示器。首先利用 YOLO 预测目标框,随后在框内执行亮度梯度分析:
\begin{equation}
    P_{pos} = \arg\max_{x,y \in ROI} I(x,y)
\end{equation}
通过选取区域最亮点作为正向提示,并结合框外负向采样,为 SAM2 提供高确定性的初始引导。

\subsection{引入全变分约束的复合损失函数}
这是本文的核心创新点。在微调阶段,我们定义了如下复合损失函数 $L_{total}$:
\begin{equation}
    L_{total} = \alpha L_{BCE} + \beta L_{Dice} + \gamma L_{TV}
\end{equation}
其中,$L_{TV}$(Total Variation Loss)专门用于惩罚预测掩码梯度的不连续性:
\begin{equation}
    L_{TV} = \sum_{i,j} \sqrt{(p_{i+1,j}-p_{i,j})^2 + (p_{i,j+1}-p_{i,j})^2}
\end{equation}
这强制模型在训练过程中学习生成符合生物解剖特征的平滑边缘。

\subsection{基于骨架化的舌中线提取}
针对语音分析需求,本文在分割掩码基础上实施了后处理:先通过形态学骨架化(Skeletonization)压缩区域,再利用 B-Spline 三次样条插值进行重采样,从而获得平滑且连续的舌面几何中心线。

\section{实验与讨论}
\subsection{实验设置}
本研究基于 PyTorch 框架,在双 NVIDIA GeForce RTX 4090 环境下完成。采用 AdamW 优化器,学习率设定为 $5 \times 10^{-6}$,累计训练 2000 个 steps。

\subsection{结果分析}
通过消融实验验证,引入 TV Loss 后,模型在保持高 IoU 的同时,掩码边缘的曲率震荡降低了约 30\%。在处理长视频时,利用 SAM2 的 \texttt{mask\_input} 传递 Logits,有效地解决了由于舌头快速移动导致的丢帧问题。

\section{结论和未来工作}
本文成功将改进的 SAM2 模型应用于复杂的超声舌面追踪任务,证明了“大模型微调 + 物理先验约束”的有效性。未来的研究将探索轻量化部署方案,以期在移动端超声设备上实现实时在线分析。

\section{参考文献}
\begin{itemize}
    \item [1] Ravi N, et al. SAM 2: Segment Anything in Images and Videos. arXiv:2408.00714, 2024.
    \item [2] Kirillov A, et al. Segment Anything. ICCV, 2023.
    \item [3] Ronneberger O, et al. U-Net: Convolutional Networks for Biomedical Image Segmentation. MICCAI, 2015.
\end{itemize}

\end{document}